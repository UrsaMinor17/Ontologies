\documentclass[10pt,a4paper]{article}
\usepackage[utf8]{inputenc}
\usepackage{amsmath}
\usepackage{amsfonts}
\usepackage{amssymb}

\author{Yasar Mahomed Abbas, Franziska Pannach, Yuvika Singh, \\Danielle Rsomething}
\title{Report Ontologies and Knowledge Bases}

\begin{document}
\maketitle


\section{Objective and Motivation}
Ontology of Folk Tales 
\section{Domain}
	\subsection{Propp Functions} 
	Russian folklorist Vladimir Propp introduced 31 invariant functions describing the morphology of the Russian magic folk tale. In his ground breaking 1928 work 'Morphology of the Folk tale', he argues that the narrative of folk tales always follows the same pattern. The narrative functions and the \textit{Dramatis Personae} (agents in the story) he introduced are strictly defined and specify recurrent units from which the tales are constructed. 
	Propp set three axioms: 
	
	\begin{itemize}
		\item Not all functions appear in every tale, but they always appear in the same order. 
		\item 	I forgot
		
		\item I forgot as well 
	
	\end{itemize}
The high formalism of this structuralism allows something as complex and highly emotional as the folk tale to be pressed in a strict pattern. Thus, they can be further used in automatically processing or when generating new tales. In Computational Linguistics, Propp's functions are used in various ways, such as for automatic markup, classification and annotation (Malec 2010), or as a foundation of an independent XML dialect (Malec 2001,  Lendvai et al. 2010). 
	His approach is still widely used not only in folk tale research but also applied to contemporary work such as the Star Wars Trilogy\footnote{http://jaced.com/2013/02/06/vladimir-propp-science-of-the-fairy-tale/}.   

\section{Conceptualization}
	\subsection{Existing Work} 
	Declerck et al. 2017 have introduced an integrated ontology that combined the ATU and the TMI motifs in a complex way. They suggested extending the ontology by including 
	
	\begin{itemize}
		\item Adding more fairy tales that fall into one of the ATU classes
		\item Adding more tales from specific collections
		\item Add Proppian functions
		
	\end{itemize}	  
	
	Aim of this project is to fulfill these three aspects. For the time being, we will concentrate on Ontology anthologies from the South African context. Our ontology will be independent but can be easily added to the existing work once it's published by Declerck et al. 
\section{Implementation}
\section{Outview}
\section{Bibliography}

\end{document}